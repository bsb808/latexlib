%%%%%%%%%%%%%%%%%%%%%%%%%%%%%%%%%%%%%%%%%%%%%%%
% latex_template_ieee_bbing.tex
%
% 28.02.2008  bsb Created 
%%%%%%%%%%%%%%%%%%%%%%%%%%%%%%%%%%%%%%%%%%%%%%%%%

% Final - 2 column style
%\documentclass[10pt,final,journal]{../latexlib/latex_ieee/IEEEtran}
% Draft - single column style
\documentclass[11pt,draftcls,journal,onecolumn]{../latexlib/latex_ieee/IEEEtran}


\usepackage{indentfirst}

% From thesis main (bbing)
\usepackage{times}
\usepackage{pdfpages}
\usepackage{fullpage}
%\usepackage{hyperref}  % Seems to break things
\usepackage{fancyhdr}
\usepackage{tabularx}
%\usepackage{enumitem} % also breaks things
\usepackage{cite}     
\usepackage{graphicx} 
\usepackage{url}      
\usepackage{amssymb,longtable,dcolumn}
\usepackage{bm}
%\usepackage{subfigure}
%\usepackage{stfloats}  % Written by Sigitas Tolusis
\usepackage{subcaption}
\usepackage[caption=false,font=footnotesize]{subfig}
%\usepackage{fixltx2e}
\usepackage{colortbl}
\usepackage{multirow}
\usepackage{amsmath}
\usepackage{units}
\usepackage{acronym}
\usepackage{csvsimple}
\usepackage{color,soul}
\DeclareRobustCommand{\hlr}[1]{{\sethlcolor{red}\hl{#1}}}
\DeclareRobustCommand{\hlg}[1]{{\sethlcolor{green}\hl{#1}}}
\DeclareRobustCommand{\hlb}[1]{{\sethlcolor{blue}\hl{#1}}}
\DeclareRobustCommand{\hly}[1]{{\sethlcolor{yellow}\hl{#1}}}

\usepackage{./latex_cmds/math_bbing}
\usepackage{./latex_cmds/my_acronyms}

\newenvironment{xitemize}{\begin{itemize}\addtolength{\itemsep}{-0.75em}}{\end{itemize}}
\newenvironment{tasklist}{\begin{enumerate}[label=\textbf{\thesubsubsection-\arabic*},ref=\thesubsubsection-\arabic*,leftmargin=*]}{\end{enumerate}}
\newcommand\todo[1]{{\bf TODO: #1}}
\setcounter{tocdepth}{2}
\setcounter{secnumdepth}{4}


\begin{document}

\newtheorem{remark}{Remark}
\renewcommand{\theremark}{\unskip}

% List of new commands
% bbing 06-11-98

%\newcommand{\spacing}[1]{\renewcommand{\baselinestretch}{#1} \normalsize}

% Degree symbol within text
\newcommand{\degrees}{$^{\circ}$}
\newcommand{\CRlb}{Cram\'{e}r Rao }

\newcommand{\btab}{\begin{tabular}}
\newcommand{\etab}{\end{tabular}}

\newcommand{\bfig}{\begin{figure}}
\newcommand{\efig}{\end{figure}}

\newcommand{\beqn}{\begin{equation}}
\newcommand{\eeqn}{\end{equation}}
\newcommand{\bdm}{\begin{displaymath}}
\newcommand{\edm}{\end{displaymath}}

\newcommand{\bearray}{\begin{eqnarray}}
\newcommand{\eearray}{\end{eqnarray}}
\newcommand{\Bearray}{\begin{eqnarray}}
\newcommand{\Eearray}{\end{eqnarray}}

\newcommand{\bgat}{\begin{gather}}
\newcommand{\egat}{\end{gather}}

\newcommand{\Htwo}{$\mathcal{H}_2$\ }
\newcommand{\Hinf}{$\mathcal{H}_{\infty}$\ }

%\newcommand{\V}[1]{\mathbf{#1}}
\newcommand{\plusminus}{{\textstyle \frac{+}{-}}}

\newcommand{\TM}{$^{TM}$}

% Definitions  %from homero!
%
%\newcommand{\mtitle}[1]{\begin{center} {\huge \bf{#1}}\end{center}}
\newcommand{\lb}{\linebreak}
\newcommand{\xaxis}{\mbox{$x$-axis}}
\newcommand{\yaxis}{\mbox{$y$-axis}}
\newcommand{\zaxis}{\mbox{$z$-axis}}
\newcommand{\bc}{\begin{center}}
\newcommand{\ec}{\end{center}}
%\newcommand{\be}{\begin{equation}}
%\newcommand{\ee}{\end{equation}}
\newcommand{\bd}{\begin{displaymath}}
\newcommand{\ed}{\end{displaymath}}
\newcommand{\bi}{\begin{itemize}}
\newcommand{\ei}{\end{itemize}}
%\newcommand{\bt}{\begin{tabular}}
%\newcommand{\et}{\end{tabular}}
\newcommand{\ba}{\begin{array}}
\newcommand{\ea}{\end{array}}
\newcommand{\baa}[2]{\begin{array}[t]{cc}
\left[#1\right.\!\!&\!\!\left.#2\right]\end{array}} 
\newcommand{\bea}{\begin{eqnarray}}
\newcommand{\eea}{\end{eqnarray}}
\newcommand{\bsea}{\begin{subeqnarray}}
\newcommand{\esea}{\end{subeqnarray}}
\newcommand{\degr}{\mbox{$^{\circ}$}}
\newcommand{\jw}{j\omega}
\newcommand{\dw}{\mbox{\rm d}\omega}
\newcommand{\dx}[1]{\,\mbox{\rm d}#1}
\newcommand{\til}{^{\sim}}
\newcommand{\abcd}[4]{\left[\begin{array}{c|c}#1&#2\\ \hline #3&#4 
\end{array}\right]}
\newcommand{\etal}{{\em et al.}}
\newcommand{\etc}{{\em etc.}}
\newcommand{\eg}{{\em e.g., }}
\newcommand{\ie}{{\em i.e., }}
\newcommand{\eqnref}[1]{Equation~(\ref{#1})}
\newcommand{\eqrefa}[1]{Eq'n~(\ref{#1})}
\newcommand{\eqrefb}[1]{(\ref{#1})}
\newcommand{\expec}[1]{\left\langle #1 \right\rangle}
\newcommand{\pder}[2]{\frac{\partial #1}{\partial #2}}
\newcommand{\half}{{\textstyle \frac{1}{2}}}
\newcommand{\msp}{\mbox{\hspace{.5in}}}
\newcommand{\ub}[2]{\underbrace{#1}_{#2}}
\newcommand{\eqn}[1]{Eq.~\ref{eq:#1}}
%
%\newenvironment{proof}{\par\noindent{\bf Proof: }}{\hfill $\Box$}
%\newenvironment{remark}{{\bf Remark: }}{}
%
\def\sinc{\mathop{\rm sinc}\nolimits}
%
\newcommand{\vdashes}[2]
   {{\vfuzz #1 \vbox to 0pt{\vskip -#2
   \vbox{\xleaders\vbox{\vskip 1pt\hfuzz 1pt\hbox to 0pt{\vrule height 4pt
   depth 0pt}\vskip 1pt}\vskip #1}}}}


\newcommand{\bmat}[1]{\left[ \begin{array}{#1}}
\newcommand{\emat}{\end{array} \right]}
%\newcommand{\bvec}{\left( \begin{array}{c}}
\newcommand{\evec}{\end{array} \right)}

  % shortcuts to thesis stuff
% Stolen from Austratlian Center for Field Robotics
% Thanks Alex!
% bbing 24.02.03

% general global definitions
\newcommand{\Def}{\ {\buildrel \triangle\over =}\ }
%\newcommand{\beq} {\begin{equation}}
%\newcommand{\eeq} {\end{equation}}
%\newcommand{\beqn} {\begin{eqnarray}}
%\newcommand{\eeqn} {\end{eqnarray}}
%\newcommand{\E}[1] {\mbox{$ {\rm E} \{ #1 \}$ }}
\newcommand{\Es}[2] {\mbox{$ {\rm E}^{#1} \{ #2 \}$ }}
\newcommand{\Set}[1] {\mbox{$ \{ #1 \} $ }}
\newcommand{\Cal}[1] {\mbox{$ {\cal #1 } $}}
\newcommand{\PR}[1]  {\mbox{$ P(#1) $}}
\newcommand{\Pri}[2]  {\mbox{$ P_{#1}(#2) $}}
\newcommand{\PRi}[2]  {\mbox{$ P_{#1}(#2) $}}
\newcommand{\Pris}[3]  {\mbox{$ P_{#1}^{#2}(#3) $}}
\newcommand{\like}[1]  {\mbox{$ \Lambda(\bf #1) $}}
\newcommand{\likei}[2]  {\mbox{$ \Lambda_{#2}(\bf #1) $}}
\newcommand{\LL}[1]  {\mbox{$ l(#1) $}} % loglikelihood
\newcommand{\LLi}[2]  {\mbox{$ l_{#1}(#2) $}} %loglikelihood
\newcommand{\EN}[1]  {\mbox{$ H(#1) $}} % entropy
\newcommand{\ENi}[2]  {\mbox{$ H_{#1}(#2) $}} % entropy
\newcommand{\mEN}[1]  {\mbox{$ \overline{H}(#1) $}} % mean entropy
\newcommand{\MI}[1]  {\mbox{$ I(#1) $}}  %mutual information
\newcommand{\est}[1]  {\mbox{$\hat{\bf #1}$}}
\newcommand{\estk}[2]  {\mbox{$\hat{\bf #1}(#2)$}}
\newcommand{\D}[1]    {\mbox{${\rm d} {#1}$}}
\newcommand{\mean}[1] {\mbox{$\overline{ #1}$}}
\newcommand{\Det}[1] {\mbox{$\mid {#1} \mid $}}
\newcommand{\One}      {\mbox{${\bf 1}$}}
\newcommand{\Zero}      {\mbox{${\bf 0}$}}
\newcommand{\grad}[1] {\mbox{${\bf\nabla} #1$}}
\newcommand{\J}[3] {\mbox{${\bf\nabla}{\bf #1}_{\bf #2}(#3)$}}
\newcommand{\Jt}[3] {\mbox{${\bf\nabla}^T{\bf #1}_{\bf #2}(#3)$}}
\newcommand{\pdf}{{\it pdf\ }}
\newcommand{\dxt}[2]  {\mbox{$\dot{\bf #1}( #2)$}}
% defining different types of vectors
% first those with no time subscripts
\newcommand{\V}[1] {\mbox{${\bf #1}$}}
\newcommand{\Vt}[1] {\mbox{${\bf #1}^T$}}
\newcommand{\Vin}[1] {\mbox{${\bf #1}^{-1}$}}
\newcommand{\Vgin}[1] {\mbox{${\bf #1}^{\dagger}$}}
\newcommand{\Vi}[2] {\mbox{${\bf #1}_{#2}$}}
\newcommand{\Vs}[2] {\mbox{${\bf #1}^{#2}$}}
\newcommand{\Vis}[3] {\mbox{${\bf #1}_{#2}^{#3}$}}
\newcommand{\Vit}[2] {\mbox{${\bf #1}_{#2}^T$}}
\newcommand{\Vini}[2] {\mbox{${\bf #1}_{#2}^{-1}$}}
\newcommand{\Vgini}[2] {\mbox{${\bf #1}^{\dagger}$}}

% next those with time subscript k (very common)
\newcommand{\Vk}[1] {\mbox{${\bf #1}(k)$}}
\newcommand{\Vkt}[1] {\mbox{${\bf #1}^T(k)$}}
\newcommand{\Vkin}[1] {\mbox{${\bf #1}^{-1}(k)$}}
\newcommand{\Vkgin}[1] {\mbox{${\bf #1}^{\dagger}(k)$}}
\newcommand{\Vki}[2] {\mbox{${\bf #1}_{#2}(k)$}}
\newcommand{\Vks}[2] {\mbox{${\bf #1}^{#2}(k)$}}
\newcommand{\Vkis}[3] {\mbox{${\bf #1}_{#2}^{#3}(k)$}}
\newcommand{\Vkit}[2] {\mbox{${\bf #1}_{#2}^T(k)$}}
\newcommand{\Vkini}[2] {\mbox{${\bf #1}_{#2}^{-1}(k)$}}
\newcommand{\Vkgini}[2] {\mbox{${\bf #1}^{\dagger}_{#2}(k)$}}
\newcommand{\tVk}[1] {\mbox{$\tilde{\bf #1}(k)$}}
\newcommand{\tVki}[2] {\mbox{$\tilde{\bf #1}_{#2}(k)$}}

% now those with general purpose time subscripts
%\newcommand{\Vec}[2] {\mbox{${\bf #1}(#2)$}}
\newcommand{\Vect}[2] {\mbox{${\bf #1}^T(#2)$}}
\newcommand{\Vecin}[2] {\mbox{${\bf #1}^{-1}(#2)$}}
\newcommand{\Veci}[3] {\mbox{${\bf #1}_{#2}(#3)$}}
\newcommand{\Vecit}[3] {\mbox{${\bf #1}_{#2}^T(#3)$}}
\newcommand{\Vecini}[3] {\mbox{${\bf #1}_{#2}^{-1}(#3)$}}
\newcommand{\Vecgin}[2] {\mbox{${\bf #1}^{\dagger}(#2)$}}
\newcommand{\Vecgini}[3] {\mbox{${\bf #1}^{\dagger}_{#2}(#3)$}}

% special symbols used very commonly
% state estimates of different sorts
\newcommand{\x}[2] {\mbox{$\hat{\bf x}( #1 \mid #2)$}}
\newcommand{\ix}[3] {\mbox{$\hat{\bf x}_{#1}( #2 \mid #3)$}}
\newcommand{\tx}[2] {\mbox{$\tilde{\bf x}( #1 \mid #2 )$}}
\newcommand{\txi}[3] {\mbox{$\tilde{\bf x}_{#1}( #2 \mid #3 )$}}
\newcommand{\z}[2] {\mbox{$\hat{\bf z}( #1 \mid #2)$}}
\newcommand{\tz}[2] {\mbox{$\tilde{\bf z}( #1 \mid #2)$}}
\newcommand{\tzt}[2] {\mbox{$\tilde{\bf z}^T( #1 \mid #2)$}}
\newcommand{\zi}[3] {\mbox{$\hat{\bf z}_{#1}( #2 \mid #3)$}}
\newcommand{\di}[3] {\mbox{$\hat{\bf \delta}_{#1}( #2 \mid #3)$}}

% variances of different sorts
\newcommand{\var}[2] {\mbox{${\bf P}( #1 \mid #2)$}}
\newcommand{\varin}[2] {\mbox{${\bf P}^{-1}( #1 \mid #2)$}}
\newcommand{\tvar}[2] {\mbox{$\tilde{\bf P}( #1 \mid #2)$}}
\newcommand{\tvarin}[2] {\mbox{$\tilde{\bf P}^{-1}( #1 \mid #2)$}}
\newcommand{\vari}[3] {\mbox{${\bf P}_{#1}( #2 \mid #3)$}}
\newcommand{\varini}[3] {\mbox{${\bf P}^{-1}_{#1}( #2 \mid #3)$}}
\newcommand{\tvari}[3] {\mbox{$\tilde{\bf P}_{#1}( #2 \mid #3)$}}
\newcommand{\tvarini}[3] {\mbox{$\tilde{\bf P}^{-1}_{#1}( #2 \mid #3)$}}

% information states and variances
\newcommand{\y}[2] {\mbox{$\hat{\bf y}( #1 \mid #2)$}}
\newcommand{\ty}[2] {\mbox{$\tilde{\bf y}( #1 \mid #2)$}}
\newcommand{\yi}[3] {\mbox{$\hat{\bf y}_{#1}( #2 \mid #3)$}}
\newcommand{\tyi}[3] {\mbox{$\tilde{\bf y}_{#1}( #2 \mid #3)$}}
\newcommand{\Y}[2] {\mbox{${\bf Y}( #1 \mid #2)$}}
\newcommand{\Yin}[2] {\mbox{${\bf Y}^{-1}( #1 \mid #2)$}}
\newcommand{\tY}[2] {\mbox{$\tilde{\bf Y}( #1 \mid #2)$}}
\newcommand{\Yi}[3] {\mbox{${\bf Y}_{#1}( #2 \mid #3)$}}
\newcommand{\Yini}[3] {\mbox{${\bf Y}_{#1}^{-1}( #2 \mid #3)$}}
\newcommand{\tYi}[3] {\mbox{$\tilde{\bf Y}_{#1}( #2 \mid #3)$}}
\newcommand{\info}[1] {\mbox{${\bf i}( #1)$}}
\newcommand{\Info}[1] {\mbox{${\bf I}( #1)$}}
\newcommand{\infoi}[2] {\mbox{${\bf i}_{#1}( #2)$}}
\newcommand{\infois}[3] {\mbox{${\bf i}_{#1}^{#2}(#3)$}}
\newcommand{\Infoi}[2] {\mbox{${\bf I}_{#1}( #2)$}}
\newcommand{\tInfoi}[2] {\mbox{$\tilde{\bf I}_{#1}( #2)$}}
\newcommand{\Infoini}[2] {\mbox{${\bf I}^{\dagger}_{#1}( #2)$}}
\newcommand{\Prop}[2] {\mbox{${\bf L}( #1 \mid #2)$}}
\newcommand{\Propi}[3] {\mbox{${\bf L}_{#1}( #2 \mid #3)$}}
\newcommand{\Z}[2] {\mbox{${\cal Z}^{#1}_{#2}$}}

% spurious ones
\newcommand{\maybe}{\ {\buildrel ?\over =}\ } %chapter 4
\newcommand{\svd} {\mbox{$\dagger$}} % chapter 4
\newcommand{\dnoise} {\mbox{$\delta d$}} % chapter 6 and 7
\newcommand{\unoise} {\mbox{$\delta u$}} % chapter 6
\newcommand{\ns}[1] {\mbox{$ #1$}} % chapter 6
\newcommand{\vs} {\vspace{0.17in}}
\newcommand{\svs} {\vspace{0.17cm}}
\newcommand{\vsf} {\vspace{0.4in}}
\newcommand{\vsff} {\vspace{1in}}
\newcommand{\veqns} {\vspace{-0.15in}}
\newcommand{\veqn} {\vspace{-0.12in}}
\newcommand{\sveqn} {\vspace{-0.06in}}
%\newcommand{\bc}{\begin{center}}
%\newcommand{\ec}{\end{center}}
%\newcommand{\bi}{\begin{itemize}}
%\newcommand{\ei}{\end{itemize}}
%\newcommand{\be}{\begin{enumerate}}
%\newcommand{\ee}{\end{enumerate}}
\newcommand{\Quote}{\parbox[t]{12.5cm}}
\newtheorem{example}{Example}


% set the figure default size
\newcommand{\SF}{0.2}
\newcommand{\SFb}{0.45}
\newcommand{\SFPic}{0.45}
\newcommand{\SFPlot}{0.45}
\newcommand{\SFc}{0.25}
% Just a lazy way of setting the figure width (percentage of text width)
% 0.7 works well for 1 column
% 0.4 works well for 2 column
\newcommand{\FigWidth}{\SFb}

% Use this one for the draft version
\newcommand{\scaleOneTwo}[2] {\scalebox{#1}}
% Use this one for the two column version
%\newcommand{\scaleOneTwo}[2] {\scalebox{#2}}

% Graphics for this paper
\graphicspath{{images/}}


% paper title
%\title{Towards an Experimentally Validated Plume Model to Support Robotic Plume Characterization}
\title{Article Example}

% author names and affiliations
% use a multiple column layout for up to three different
% affiliations
\author{Brian~Bingham$^{1}$% <-this % stops a space
\thanks{$^{1}$ Brian~Bingham is with the Department of Mechanical and Aerospace Engineering, Naval Postgraduate School, Monterey, CA 93950, USA. {\tt\small bbingham@nps.edu}}%
}

% make the title area
\maketitle

%\begin{abstract}
%Abstract
%\end{abstract}
% no keywords

\IEEEpeerreviewmaketitle

\section{Introduction}
Lorem ipsum dolor sit amet, consectetur adipiscing elit. Pellentesque sit amet posuere mi. Etiam pretium ipsum massa. Curabitur dignissim quam in semper suscipit. Morbi sodales diam consequat ex euismod cursus \cite{bingham03phd}.

\section{Background}
 Curabitur cursus ligula eget nisi fermentum imperdiet. Nam commodo massa eget nisl lobortis, vel consequat sem gravida. Sed turpis lorem, aliquet sed ultricies eu, pharetra consectetur odio. Curabitur id iaculis dolor. 

\section{Simplified Models for Control}
The horizontal-plane maneuvering model uses the state vector $\bm{\nu}=[u,v,r]^T$ where the velocities $u$, $v$ and $r$ are in the surge, sway and yaw directions respectively. 

Euler angles for angular deplacements/velocities are
\begin{itemize}
\item roll $\phi$, $p=\dot{\phi}$
\item pitch $\theta$, $q=\dot{\theta}$
\item yaw $\psi$, $r=\dot{\psi}$
\end{itemize}

\section{Maneuvering Models}
\subsection{Nonlinear Maneuvering Model based on Second-order Modulus Functions}
\beqn
\underbrace{\bm{M}_{RB}\dot{\bm{\nu}}+\bm{C}_{RB}(\bm{\nu})\bm{\nu}}_\text{rigid-body forces} +
\underbrace{\bm{M}_A\dot{\bm{\nu}}_r + \bm{C}_A(\bm{\nu}_r)\bm{\nu}_r + 
\bm{D}(\bm{\nu}_r)\bm{\nu}_r}_\text{hydrodynamic forces}
= \bm{\tau}+\bm{\tau}_{wind}+\bm{\tau}_{waves}
\label{e:fossenmodel}
\eeqn
where $\bm{\nu}_r$ is the velocity vector relative to an irrotational water current $\bm{\nu}_c$, i.e., $\bm{\nu}=\bm{\nu}_r+\bm{\nu}_c$.  The rigid body kinetics are represented by the rigid body mass $\bm{M}_{RB}$ 
\beqn
\bm{M}_{RB}= \left[ 
\begin{array}{ccc}
m & 0 & 0 \\
0 & m & m x_g \\
0 & m x_g & I_z 
\end{array} \right],
\eeqn
where $m$ is the mass of the vehicle, $I_z$ is the moment of inertia about the body-centered z-axis and $x_g$ is distance along the x-axis from the origin of the body-centered frame to the center of gravity of the vessel.  The second rigid body term is the Coriolis-centripetal matrix,
\beqn
\bm{C}_{RB}(\bm{\nu})= \left[ 
\begin{array}{ccc}
0 & 0 & -m(x_gr+v) \\
0 & 0 & mu \\
m(x_gr+v) & -mu  & 0 
\end{array} \right].
\eeqn
Equivalently we can express the same model in non-matrix form, where the speed equation in the surge direction is 
\beqn
\underbrace{m \dot{u}}_\text{RB inertia}  
- \underbrace{m x_g r^2}_\text{RB centripetal}
- \underbrace{mvr}_\text{RB Coriolis}
=
\underbrace{X_{\dot{u}} \dot{u}_r}_\text{AM inertia}
+ \underbrace{Y_{\dot{v}}v_r r}_\text{AM Coriolis}
+ \underbrace{Y_{\dot{r}}r^2}_\text{AM centripetal}
+ \underbrace{X_u u + X_{u|u|}|u|u}_\text{Drag} 
+ \underbrace{\tau}_\text{Thrust}
\label{e:fullu}
\eeqn


\hl{What/Why is the quadratic yaw drag expressed as $N_{v|v|}|v|v$?  Seems like it would be $N_{r|r|}|r|r$.}


\section{Graphics Examples}

\begin{figure}[h]
  \centering
  \includegraphics[width=0.7\textwidth]{example.png}
  \caption{System architecture of the Gazebo robot simulator.}
  \label{fig:gazebo-arch}
\end{figure}



\begin{figure}[h]
  \centering
  \begin{subfigure}[t]{0.3\textwidth}
    \includegraphics[width=\textwidth]{example.png}
    \caption{The Boston Dynamics Atlas humanoid robot simulated in Gazebo for
             the DARPA Robotics Challenge (DRC)}
  \end{subfigure}
  ~
  \begin{subfigure}[t]{0.3\textwidth}
    \includegraphics[width=\textwidth]{example.png}
    \caption{The NASA R5 humanoid robot simulated in Gazebo for the Space
             Robotics Challenge (SRC)}
  \end{subfigure}
  ~
  \begin{subfigure}[t]{0.3\textwidth}
    \includegraphics[width=\textwidth]{example.png}
    \caption{The Johns Hopkins prosthetic limb simulated in Gazebo for the
             DARPA HAPTIX program.}
  \end{subfigure}\\
  ~
  \begin{subfigure}[t]{0.3\textwidth}
    \includegraphics[width=\textwidth]{example.png}
    \caption{An industrial robot simulated for the NIST ARIAC program.}
  \end{subfigure}
  ~
  \begin{subfigure}[t]{0.3\textwidth}
    \includegraphics[width=\textwidth]{example.png}
    \caption{A large swarm of unmanned aerial vehicles (UAVs) simulated in
             Gazebo for the DARPA SASC program.}
  \end{subfigure}
  ~
  \caption{Examples of Gazebo used to simulate a variety of robots in
           government-sponsored research \& development programs.}
  \label{fig:gazebo_examples}
\end{figure}

\clearpage

That's all folks...

% standard IEEE bibliography style from:
% http://www.ctan.org/tex-archive/macros/latex/contrib/supported/IEEEtran/bibtex
\bibliographystyle{./ieee/IEEEtran}
%\bibliographystyle{apalike}
% argument is your BibTeX string definitions and bibliography database(s)
\bibliography{./bibtex/bbing_master}
%\bibliography{./latex_ieee/IEEEabrv}

% if you will not have a photo at all:

% that's all folks


\end{document}


